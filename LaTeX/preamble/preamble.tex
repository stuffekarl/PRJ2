% Dokumentklassen sættes til memoir.
% Manual: http://ctan.org/tex-archive/macros/latex/contrib/memoir/memman.pdf
\documentclass[a4paper,11pt,oneside,article]{memoir}
\setlrmarginsandblock{*}{2.5cm}{0.75} % højre og venstre 
\setulmarginsandblock{3cm}{*}{1.2} % top og bund 
\checkandfixthelayout[nearest] % specifikt valg af højde algoritme
 
% Danske udtryk (fx figur og tabel) samt dansk orddeling og fonte med
% danske tegn. Hvis LaTeX brokker sig over æ, ø og å skal du udskifte
% "utf8" med "latin1" eller "applemac". 
\usepackage[utf8]{inputenc}
\usepackage[danish]{babel}
\usepackage[T1]{fontenc}
\usepackage{mflogo}

%sexy pdf'er
%\usepackage[export]{adjustbox}
\usepackage{pdfpages}
\usepackage{pdflscape}

%Kompakte lister
\usepackage{paralist}
 
% Matematisk udtryk, fede symboler, theoremer og fancy ting (fx kædebrøker)
\usepackage{amsmath,amssymb}
\usepackage{bm}
\usepackage{amsthm}
\usepackage{mathtools}
\parindent=0pt 

% Fancy ting med enheder og datatabeller. Læs manualen til pakken
% Manual: http://www.ctan.org/tex-archive/macros/latex/contrib/siunitx/siunitx.pdf
\usepackage{siunitx}
 
% Indsættelse af grafik.
\usepackage{graphicx} 
\usepackage{fix-cm} 
\usepackage{soul}
\sodef\an{}{0.13em}{0em}{0em} \sodef\ann{}{0.13em}{0.5em}{0em}
 
%Fancy tabeller.
%\usepackage[table]{xcolor}
\usepackage{multirow}
\usepackage{rotating} %sidewaystables!
\usepackage{longtable} %tables spanning multible pages.
\usepackage{tablefootnote} %for at indstætte fornoter i tabeller.
\usepackage{hhline} %Fixer farvede felter
\usepackage{ltxtable} %Longtabular X
\usepackage{tabularx} %Med dynamisk bredte

%URL fodnoter
\usepackage{url}

% Reaktionsskemaer. Læs manualen for at se eksempler.
% Manual: http://www.ctan.org/tex-archive/macros/latex/contrib/mhchem/mhchem.pdf
\usepackage[version=3]{mhchem}

%Lav chapter clickable og fjern border
\usepackage{hyperref}
\hypersetup{
    colorlinks,
    citecolor=black,
    filecolor=black,
    linkcolor=black,
    urlcolor=black
}

%Table of contents settings
\setsecnumdepth{subsection} % organisational level that receives a numbers
\settocdepth{subsection}   % print table of  for level 3

%Til programkode
\usepackage{listings}
\usepackage{color}

\definecolor{dkgreen}{rgb}{0,0.6,0}
\definecolor{gray}{rgb}{0.5,0.5,0.5}
\definecolor{mauve}{rgb}{0.58,0,0.82}
 
\lstset{ 
  language=C++,                % the language of the code
  basicstyle=\footnotesize,           % the size of the fonts that are used for the code
  numbers=left,                   % where to put the line-numbers
  numberstyle=\tiny\color{gray},  % the style that is used for the line-numbers
  stepnumber=1,                   % the step between two line-numbers. If it's 1, each line 
                                  % will be numbered
  numbersep=5pt,                  % how far the line-numbers are from the code
  backgroundcolor=\color{white},      % choose the background color. You must add \usepackage{color}
  showspaces=false,               % show spaces adding particular underscores
  showstringspaces=false,         % underline spaces within strings
  showtabs=false,                 % show tabs within strings adding particular underscores
  frame=single,                   % adds a frame around the code
  rulecolor=\color{black},        % if not set, the frame-color may be changed on line-breaks within not-black text (e.g. commens (green here))
  tabsize=2,                      % sets default tabsize to 2 spaces
  captionpos=b,                   % sets the caption-position to bottom
  breaklines=true,                % sets automatic line breaking
  breakatwhitespace=false,        % sets if automatic breaks should only happen at whitespace
  title=\lstname,                   % show the filename of files included with \lstinputlisting;
                                  % also try caption instead of title
  keywordstyle=\color{blue},          % keyword style
  commentstyle=\color{dkgreen},       % comment style
  stringstyle=\color{mauve},         % string literal style
  escapeinside={\%*}{*)},            % if you want to add LaTeX within your code
  morekeywords={*,...}               % if you want to add more keywords to the set
}

%Til at udregne forskel mellem sider, brug \pagedifference{A}{B} mellem to labels A og B.
\usepackage{refcount}
\newcommand{\pagedifference}[2]{%
  \number\numexpr\getpagerefnumber{#2}-\getpagerefnumber{#1}\relax}
 
%Til at lave referencer med:
\usepackage{cite}

%Til at lave eksterne \ref til \labels
\usepackage{xr}

%Forsøg på nice lister i tabeller
\usepackage[shortlabels]{enumitem}

\newenvironment{packed_enum}{
\begin{enumerate}[1., topsep=0pt, nosep, partopsep=0pt, parskip=0pt, itemsep=0pt, parsep=0pt]
}{\end{enumerate}}

\newenvironment{packed_item}{
\begin{itemize}[•, topsep=0pt, nosep, partopsep=0pt, itemsep=0pt, parsep=0pt]
}{\end{itemize}}

