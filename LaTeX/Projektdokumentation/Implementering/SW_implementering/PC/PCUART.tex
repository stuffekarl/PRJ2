\subsection{PC UART (Kasper)}

PC'ens UART har til formål at oversætte de data, der er i \textit{Scenario}-objektet, og transmittere dem over på transmitteren. Dette gøres ved brug af et open-source bibliotek kaldet RS232\cite{lib:UART}, der følger den protokol det ønskes at transmittere data med.

For at sende data ud til Transmitteren, skal der oprettes en forbindelse gennem en USB port. her bruges \texttt{UART\_connect} til formålet.

Systemet kræver også at kunne få status på kodelåsen, hvilket metoden \texttt{getLockStatus} står for. Kravet for at bruge \texttt{getLockstatus} er, at \texttt{UART\_connect} har været kørt, for at finde en gyldig port at sende og modtage på.

Når data skal transmitteres bruges \texttt{sendScen}, der tager en reference til et scenarie som parameter, og får oversat dataene i scenariet til at følge UART protokollen, hvorefter den sender dem over til transmitteren.


\subsubsection{UART\_connect}
Metoden \texttt{UART\_connect} bruges til at finde en gyldig COM-port på computeren. Metoden kører igennem alle COM-porte på PCen ved at bruge RS232-bibliotekets funktion \texttt{RS232\_openComport}, til at tjekke om der sidder et RS232 stik i den givne COM-port. Så længe den testede COM-port ikke har et RS232 stik i, tæller metoden Cport\_nr en op, og derved tester den næste COM-port, indtil en gyldig COM-port er fundet.

\begin{lstlisting}
while (RS232_OpenComport(cport_nr, bdrate, mode))
		{

			printf("ugyldig Com-port\n");
			cport_nr++;
		}
\end{lstlisting}

\subsubsection{getLockStatus}
Metoden \texttt{getLockStatus} finder ud af kodelåsens status ved at sende et \texttt{L} ud på COM-porten. Dette sker ved at bruge RS232-bibliotekets funktion \texttt{RS232\_cputs()}. Parametrene for funktionen er COM-port nummeret \texttt{cport\_nr} og den mængde af data i form af chars, der skal sendes.\\
Da det kræver COM-port nummeret, er det derfor også nødvendigt, at metoden \texttt{UART\_connect} har været kørt, før det er muligt at bruge \texttt{getLockStatus}.

\begin{lstlisting}
RS232_cputs(cport_nr, "L");
\end{lstlisting} 

Når beskeden om at tjekke status er blevet sendt til transmitteren, begynder metoden at læse på porten og venter på et svar tilbage fra transmitteren. Svaret vil indeholde et \texttt{L} eller et \texttt{U}, afhængig af om kodelåsen er hhv. låst eller oplåst. Til aflæsning på COM-porten bruges RS232-bibliotekets funktion \texttt{RS232\_PollComport()}.
Hvis det aflæste er et \texttt{L} returneres \texttt{true}, for at kodelåsen er låst. Hvis et \texttt{U} aflæses returneres \texttt{false}, hvilket betyder kodelåsen er låst op.

\subsubsection{sendScen}
metoden Send Scen har til opgave at oversætte data fra scenariet til chars som følger UART protokolen.
til at starte med kører metoden koden nedenfor:
\begin{lstlisting}
RS232_cputs(cport_nr, "N");
\end{lstlisting} 
\texttt{N}'et betyder at transmitteren skal være klar på at der nu kommer et ny datastrøm fra PCen. 
Metoden henter herefter den første aktion, så \texttt{get} funktioner kan bruges til at hente data ud af aktionen.
Kommandoen bliver sat direkte ind, da \texttt{get} funktionen for den returnerer en char, derfor sker ingenting med den. 
Tiden skal regnes om, så transmitteren ved hvor længe der skal gå, før den skal udføre en kommando. Dette sker i koden set nedenfor:
\begin{lstlisting}
				//udregner tiden til den gaeldne commando skal ekseveres
				int timeTillExecution;
				if (action.getTime() >= Ctime())
				{
					timeTillExecution = action.getTime() - Ctime();
				}
				else
				{ 
					timeTillExecution = 1440 - (Ctime()-action.getTime()); 
				}
				//tager sig af at omregne tiden fra int til char og dele tiden op i 2 chars, da den kan ske at fylde mere end 8 bit
				int timeL1 = timeTillExecution % 16; 
				int timeL2 = int((floor(timeTillExecution / 16) % 16)
				int timeH = int((floor(timeTillExecution / 256)));

				TimeLow1 = ((char)timeL1 +48);
				TimeLow2 = ((char)timeL2+48)
				TimeHigh = ((char)timeH)+48);
\end{lstlisting} 
Integeren \texttt{timeTillexecution} bliver udregnet ud fra en af 2 udregning. Den valgte udregning afhænger af om tidspunktet allerede har passeret systems klokkeslet. Hvis dette er tilfældet køres udregningen i linje 9. Denne udregning tager hensyn til hvor mange minutter der er tilbage på dagen og kører derfor på det angivne klokkeslet næste dag. udregningen i linje 5 sker hvis klokkeslettet endnu ikke har passeret tiden, hvor kommandoen skal udføres. Den udregner hvor mange minutter der er fra det gældende klokkeslet og frem til kommandoen skal udføres.
\texttt{Ctime()} funktionen som fremgår flere steder er en hjælpefunktionen som returnerer en integer med klokkeslettet på PCen.
Når \texttt{timeTillExecution} er udregnet, skal den omsættes til Chars. Da den maksimale værdi der vil fremkomme er 1440(antallet af minutter på et døgn), skal der bruges 2 chars til at gemme dataene i.
\begin{lstlisting} 
				int timeL1 = timeTillExecution % 16; 
				int timeL2 = int((floor(timeTillExecution / 16) % 16)
				int timeH = int((floor(timeTillExecution / 256)));
\end{lstlisting} 
\texttt{timeL1} er hvor de 4 første bit gemmes, derfor tages modulus til 16 for at finde resten, og altså den værdi der skal gemmens i \texttt{timeL1}.
\texttt{TimeL2} er hvor de midterste bit gemmes, derfor nedrundes der,og denne omskrives til en int. Derefter tages modulus til 16.
\texttt{timeH} er hvor de 4 største bit gemmes. For at finde værdien der skal gemmes heri, divideres \texttt{timeTillExecution} med 256, og rundes ned, for præcision. Da nedrundingen sker som en double omskrives det til en int, efter nedrundingen.

\texttt{timeL1}, \texttt{timeL2} og \texttt{timeH} har efter udregningen en makimalstørrelse på 16, og kan derfor laves om til en char, med ASCII kode mellem 48 og 64
\begin{lstlisting} 
				TimeLow1 = ((char)timeL1 +48);
				TimeLow2 = ((char)timeL2+48)
				TimeHigh = ((char)timeH)+48);
\end{lstlisting} 

Unit oversættes ved brug af en switch set i koden nedenfor:
\begin{lstlisting} 
switch (action.getUnit()) // tager sig af at oversaette unit til at foelge UART protocollen
			{
			case 1:
				unit = 'A';
				break;
			case 2:
				unit = 'C';
				break;
			case 3:
				unit = 'P';
				break;
			case 4:
				unit = 'N';
				break;
			default: 
				unit = '0';
				break;

			}
\end{lstlisting} 
Switchen tager uniten ind fra aktionen, og alt efter værdien sættes unit til den char der passer en den gældende unit.

UARTen sender nu dataene fra aktionen ud på COM-porten ved brug af funktionen \texttt{RS232\_cput}

\begin{lstlisting}
char data[6] = { unit, command, TimeHigh, TimeLow2, TimeLow1 }; // data der skal sendes over
			strcpy(str[c], data);
			RS232_cputs(cport_nr, str[c]);
\end{lstlisting}

når dataene er sendt på på porten, venter UARTen 100milisekunder, hvorefter den begynder forfra med aktion nr.2. Dette sker til alle 20 aktioner er overført til transmitteren, hvorefter metoden afslutter.


\subsubsection{stopAll}
metoden \texttt{stopAll} er til at slukke alle enheder på system. Dette gør metoden ved at bruge funktionen \texttt{RS232\_cput} til at sende et \texttt{S} til transmitteren. Transmitteren ved ifølge protokolen at et \texttt{S} betyder den skal slukke alt.
\begin{lstlisting}
RS232_cputs(cport_nr, "S");
\end{lstlisting}