\subsection{Scenario (Kristian S.)}

\subsubsection{Scenario constructor}

Scenario klassen bruger STL(Standard Template Library) vector container format istedet for et array. Constructoren bruger STL's assign funktion til at oprette 20 aktion objekter med default værdier.

\subsubsection{getAction}

Metoden returner en reference til et aktion objekt i vectoren på baggrund af pladsen i den.

\subsubsection{sortActions}

sortActions bruger STL's sorterings funktionalitet til at sætte de aktions objekter i numerisk rækkefølge set i forhold til deres tids værdi. For at kunne sammenligne aktion objekter, har aktion klassen fået overloadet dens "<" operator. sortAction har ikke længere en returværdi, men muterer et scenarios vector af action objekter.

\begin{lstlisting}
void Scenario::sortActions()
{
		sort(scen_.begin(), scen_.end());
}
\end{lstlisting}

\subsubsection{clearActions}

clearActions bruger clear funktionen i STL for at slette alle actions objekter, hvor efter de bliver oprettet igen med default værdier.


\subsubsection{Ostream Operator}

Scenario klassens ostream operator er blevet overloadet til brug i UI klassen. "<<" operatoren udskriver nu alle aktion objekters værdier som: Tid, kommando og enhed.


\clearpage