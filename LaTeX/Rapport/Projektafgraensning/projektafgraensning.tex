\chapter{Projektafgrænsning}
Der er som nævnt under opgaveformuleringen krav om at systemet skal indeholde:
\begin{itemize}
\item 1 PC som anvendes til konfiguration og styring
\item 1 STK500 kit som anvendes til central transmitter controller
\item 1 DE2-kodelås
\end{itemize}
Der er desuden krav om at X.10 protokollen \cite[s. 12]{lib:AN236} skal anvendes til kommunikation via el-nettet. 

Som det ligeledes er nævnt under opgaveformuleringen, anvendes et 18 VAC elnet i stedet for det almindelige 230 VAC elnet af sikkerhedsmæssige årsager. 

Til at starte med er det fuldstændige system beskrevet i projektdokumentationen, men af tidsmæssige årsager, er radio- og TV-delen udeladt fra og med signalbeskrivelser i Systemarkitekturen på side \pageref{P-tbl:signalbeskriv} i projektdokumentationen.
Det ville kræve en del mere arbejde at implementere kommunikationen til radio og TV fra de respektive receivere, da denne skulle foregå ved brug af infrarødt lys. 

Systemets hardware er implementeret på veroboard, hvilket betyder at accepttest af afskærmningen af elektriske komponenter ikke er gennemført. 

Kun dele af X.10 protokollen, som er relevante for opfyldelse af systemets mål, er anvendt. 
For mere detaljeret beskrivelse af den anvendte X.10 protokol henvises til afsnittet \textit{Protokol for X.10} på side \pageref{P-prot_x10} i projektdokumentationen.
\clearpage
