\chapter{Indledning}

Næsten alle har prøvet at have indbrud i deres hjem eller kender nogen, som har haft denne ubehagelige oplevelse. 
Der er mange forskellige måder at sikre sit hjem mod indbrud; dette kan fx ske ved opsætning af kraftige låse/døre/vinduer, vagtservice eller overvågnings- og alarmsystemer. 
Langt de fleste indbrud i et hjem sker når ingen er hjemme, der ligger derfor en oplagt mulighed for at forebygge og forhindre indbrud, ved at få indbrudstyven til at tro at der er beboere til stede i hjemmet. 

Dette projekt omhandler design og implementering af et system, som vha. allerede tilstedeværende elektriske enheder, netop kan simulere tilstedeværelse i en bolig. 
Systemet styres fra en almindelig PC, der kommunikerer med de elektriske enheder via elnettet. 
Der er således et minimalt behov for installation af hardware og ledninger i hjemmet. 

\section{Forkortelser i rapporten}
I dette dokument er der anvendt en række forkortelser:

\begin{table} [h]
	\centering
	\begin{tabular}{|l|l|}
	\hline
	\textbf{Forkortelse} & \textbf{Forklaring} \\ \hline
	DE-2 & Terasic Altera DE-2 udviklingsboard \\ \hline
	STK500-kit & Atmel developmentboard monteret med ATMega32 microcontroller \\ \hline 
	V-model & Se I2ISE Compendium \cite{lib:T-006}\\ \hline
	ASE-modellen & Se Projektoplæg \cite{lib:Projektoplaeg}\\ \hline
	Scrum & Agile framework tool  \\ \hline
	SysML & System Modeling language \cite{lib:T-006} \\ \hline
	FPGA & Field Programmable Gate Array \\ \hline
	X.10 & Protokol for kommunikation via el-nettet (Se AN236). \cite[s. 12]{lib:AN236} \\ \hline
	HW & Hardware \\ \hline
	SW & Software \\ \hline
	UC & Use Case \\ \hline
	ISE & Indledende System Engineering \\ \hline
	BDD & Blokdefinitionsdiagram \\ \hline
	IBD & Internt blokdiagram \\ \hline
	SVN & Subversion \\ \hline
	MSYS & Microcontroller Systemer \\ \hline
	cd & Class Diagram \\ \hline
	UI & User Interface \\ \hline
	UART & Universal Asynchronous Receiver/Transmitter \\ \hline
	RS232 & Recommended Standard 232, seriel kommunikation \\ \hline
	PWM & Pulse Width Modulation \\ \hline
	STL & Standard Template Library \\ \hline
	\end{tabular}
\end{table}
\clearpage
\section{Navne i dokumentationen}
I projektdokumentationen er der i alle overskrifter angivet hvem der har arbejdet med området. Der er for overskuelighedens skyld kun anvendt fornavne:

\begin{table}[h]
	\centering
	\begin{tabularx}{\textwidth-0.5cm}{|l|X|r|l|}
	\hline
	\textbf{Anvendt navn} & \textbf{Fulde navn} & \textbf{Studienummer} & \textbf{Forkortelser}\\ \hline
	Morten & Morten Hasseriis Gormsen & 201370948 & MHG \\ \hline
	Kristian T. & Kristian Thomsen & 201311478 & KT \\ \hline
	Philip & Philip Krogh-Pedersen & 201311473 & PKP \\ \hline
	Lasse & Lasse Barner Sivertsen & 201371048 & LS \\ \hline
	Henrik & Henrik Bagger Jensen & 201304157  & HB\\ \hline
	David & David Erik Jensen & 11229          & DE\\ \hline
	Kasper & Kasper Torp Samuelsen & 201311498 & KTS\\ \hline
	Kristian S. & Kristian Søgaard Sørensen &  20115255 & KS \\ \hline
	Alle & Alle 8 medlemmer af projektgruppen & - & - \\ \hline
	\end{tabularx}
\end{table}

\clearpage