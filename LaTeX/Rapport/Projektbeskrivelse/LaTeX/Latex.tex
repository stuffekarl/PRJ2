\section{LaTeX}

der er i gruppen blevet diskuteret, hvorvidt dokumentation samt rapport skulle skrives i Microsoft Word eller LaTeX\cite{lib:LaTeX}. 
Alle havde erfaringer med Word og kun 2 ud af 8 havde prøvet at arbejde i LaTeX. 
At Word dog alligevel blev fravalgt skyldes erfaringer fra sidste semester, hvor store Word filer kunne blive for uhåndterbare til redigering. 
LaTeX derimod gav muligheden for, at dokumentation kunne brydes ned i mindre .tex filer (LaTeX's filtype), og derved give et bedre overblik.
For at alle gruppemedlemmer kunne begynde at bruge LaTeX, blev der internt afholdt en øvelseslektion, hvor de basale redskaber blev gennemgået og afprøvet.
Gennem projektet var der opstået problemer, der ikke nødvendigvis ville være sket, hvis Word var blevet brugt som skriveredskab (f.eks formattering af tabeller). 
Men da LaTeX er et markup language, ligesom HTML, kan det mere præcist bestemme hvor elementer skal placeres på siden. 
Andre fordele er alle overskrifter/brødtekst/skrifttyper/skriftstørrelse er defineret i en såkaldt preamble, så alting ser ens ud, uanset hvilken .tex fil der arbejdes i, samt at interne og eksterne referencer nemt kunne styres.