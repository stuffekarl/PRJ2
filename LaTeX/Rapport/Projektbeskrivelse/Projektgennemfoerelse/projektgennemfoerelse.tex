\section{Projektgennemførelse}
Som noget af det første udarbejdede gruppen en samarbejdsaftale \cite{lib:Samarbejdsaftale}, der fastlagde rammer for gennemførelse af projektet. 

Det er klart, at selve punkterne i aftalen har haft betydning for projektarbejdet, men også processen i at lave aftalen, var særdeles givende. 
Gruppen havde ikke arbejdet sammen som helhed før, så det var vigtigt at få afstemt forventninger til arbejdet, herunder særligt ambitionsniveauet for projektet. 


Der var således fra starten enighed om at gruppens prioriterede ambitioner var:

\begin{enumerate}
\item At lære mest muligt.
\item Lave en god rapport og dokumentation.
\item Lave et komplekst fungerende system.
\end{enumerate}

Der er undervejs i arbejdet – særligt i de tidlige faser – lagt stor vægt på at arbejde så meget sammen som muligt, for at optimere fælles læring og vidensdeling. 
Meget store dele af dokumentationen er derfor skrevet i fællesskab. 
Det har medvirket til et bedre resultat, men det har også betydet, at nogle faser har taget længere tid, end hvis opgaver var blevet fordelt mellem gruppemedlemmer og efterfølgende samlet. 

Samarbejdsaftalen indeholder også rammer for gruppearbejdet, såfremt et eller flere gruppemedlemmer ikke levede op til aftalen. 
Dette har ikke været på tale, efter rammerne var fastlagt. 
Alle i gruppen har været særdeles engagerede, hvilket selvfølgelig har været meget positivt, men det har også betydet, at gruppen nogle gange har skullet bruge en del ressourcer på at finde kompromis mellem mange inputs og meninger om store eller små ting. 

Samarbejdsaftalen placerede desuden en række ansvarsområder til enkelte gruppemedlemmer. 
Der er senere i forløbet blevet fordelt flere områder:
\begin{itemize}
\item Koordinator

Morten har haft dette ansvarsområde, og dermed ansvar for mødeindkaldelser, mødereferater, føring af overordnet log, opdatering af fælles kalender, opdatering af tidsplan mm. 

\item Ordstyrer

Dette område har gået på skift mellem gruppemedlemmerne for 2 uger af gangen. 
Det har primært indebåret ordstyring under gruppemøder og fælles arbejdsmøder.

\item Dropbox ansvarlig

Kristian T. har haft dette ansvarsområde og dermed ansvar for vores fælles filstruktur til deling af dokumenter.

\item Apache Subversion

David har haft ansvar for dette værktøj, som har været anvendt til versionsstyring og deling af kildekode. 

\item Latex ansvarlig

Kristian S. og Kristian T. har haft dette område. 
Gruppen valgte på et tidligt tidspunkt at rapport og dokumentation skulle skrives med dette værktøj. 
Ansvarsområdet har indebåret afholdelse af en lektion for øvrige gruppemedlemmer og løbende hjælp til brug af værktøjet. 

\item Scrum ansvarlig

David har fungeret som Scrum ansvarlig i de faser hvor Scrum er blevet anvendt. 

\item HW ansvarlig

Morten har under design og implementering af hardware fungeret som koordinator for dette.

\item SW ansvarlig

Kasper har under design og implementering af software fungeret som koordinator for dette. 
\end{itemize}


Som tidligere nævnt har der under projektarbejdet været lagt vægt på, at hele gruppen arbejdede sammen så meget som muligt, men under design og implementeringsfaserne blev gruppen delt op i undergrupper. 
Det skete i første omgang ved at Kasper, David, Kristian T. og Kristian S. udarbejdede SW design, mens Morten, Henrik, Lasse og Philip lavede HW design. 
Under implementeringsfasen blev Morten og Kasper i deres respektive grupper, og fungerede som tovholdere for disse, mens de øvrige medlemmer i projektgruppen havde mulighed for at arbejde med implementering inden for begge områder. 

Kommunikationen internt i gruppen har foregået via en fælles Facebook gruppe og ugentlige arbejdsmøder, hvilket har fungeret godt.

Den fælles kalender på Ingeniørhøjskolens Campusnet har været anvendt til at holde styr på vejledermøder, arbejdsmøder og øvrige aftaler.
 
I starten af forløbet blev der udarbejdet en overordnet tidsplan \cite{lib:Tidsplan}, ud fra en skabelon i projektoplægget \cite{lib:Projektoplaeg} med deadlines for review mm.
Gruppens egen tidsplan blev fastlagt således, at projektarbejdet så vidt som muligt var en uge foran. 
Dette skete på baggrund af, at der så ville være tid til uforudset arbejde og skrivning af rapport. 
Det er med lidt svingende udfordring lykkes at holde denne tidsplan, hvilket har givet projektgruppen ekstra god tid, særligt i de sidste faser.

Undervejs i projektarbejdet har der været afholdt to reviews \cite{lib:Review1} \cite{lib:Review2}, hvor projektarbejdet er blevet gennemset og kommenteret af en anden gruppe og vice versa. 
Det har naturligvis betydet ekstra arbejde at skulle gennemse en anden gruppes arbejde, men kommentarene fra en anden projektgruppe og deres vejleder, samt fra denne projektgruppes egen vejleder, har været givende for det endelige resultat. 
