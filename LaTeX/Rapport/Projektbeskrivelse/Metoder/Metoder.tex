\section{Metoder}

\subsection{SysML}
SysML har givet et bedre overblik over projektet, da systemet har kunnet deles op i blokke, og det herefter var muligt at arbejde med de individuelle blokke. Ud fra disse enkelte blokke, var opdelingen af parts og ports nemt. 
BDD-diagrammer har givet overblik over kompostion af blokkene via relationer, som er specificeret i diagrammet.
IBD-diagrammer har givet mulighed for at holde styr på signaler, og kommunikationsveje mellem de forskellige blokke. Signalerne, der går imellem blokkene, gav mulighed for at lave detaljerede grænseflader på systemets elementer, og i såfald hvilken type, da det vides, om der skal forekomme kommunikation mellem disse elementer. 
Use Cases har givet mulighed for at designe det ønskede scenarie, og tage højde for de faldgrupper, der kan opstå undervejs i scenariet. Use Cases er blevet anvendt til at fremstille sekvensdiagrammer, så de stemmer overens med udførelsen af de enkelte steps i use casen.

\subsection{Scrum}
Scrum er blevet brugt til uddeling og adminstration af opgaver under projektet. Standard scrum-roller blev ikke brugt, men i stedet var der et fælles ansvar for at få lagt opgaver ind på scrum-boardet. Da der ikke var et fast grupperum til projektarbejdet, var det ikke muligt at bruge et klassisk scrum-board, og der blev derfor anvendt en webbaseret løsning i stedet  \cite{lib:Axosoft}. Der blev brugt en form for daglige Scrummøder. Alle medlemmer kunne hver især kort fremlægge, hvad de  havde lavet, og hvad planen for resten af dagen var. De mange opdateringer på fremskridtet med arbejdet, bidrog til at der var en konstant ide om, hvor langt projektgruppen var nået i den nuværende fase. Det startede ud med, at Scrum blev testet i Softwaregruppen under designfasen, da der var mange små opgaver, som skulle løses, hvilket gjorde det nemmere at holde styr på hver opgaves status, så intet blev glemt. Efter design-fasen blev alle medlemmer af gruppen introduceret til brugen af Scrum, da der i gruppen var interesse for at arbejde med både software og hardware.
På grund af relativ få opgaver i hardware-gruppen, synes flere at scrum-boardet ikke bidrog til overblik i udviklingen. Til udvikling af software har scrum-boardet været godt til at danne overblik over de mange mindre opgaver som systemet bestod af.

\subsection{V-model og ASE-model}
I projektet har V-modellen\cite{lib:Projektoplaeg} været anvendt indtil modul-designet, hvor der blev skiftet over til at bruge ASE-modellen. 
ASE-modellen var blevet brugt i fuld form i arbejdsfordelingen. Arbejdet blev delt op i software og hardware, under design og implementering. Det har gjort arbejdet mere effektivt, da det har givet den enkelte mulighed for mere fordybelse til at arbejde med et specifikt område. Efter de forskellige moduler har været igennem modultest, begyndte integrationstesten samlet, hvor hardware og software skulle samles.

\subsection{Versions-styring}
Versioner på dokumentations-dokumenter er blevet opdateret løbende med kommentarer fra vejleder og reviews. Væsentlige ændringer i design og dokumentation har givet anledning til versions-ændring, hvilket hjælper med at holde styr hvilke ændringer projektet har gennemgået.

\subsection{Review}
De reviews\cite{lib:Review1}\cite{lib:Review2}, der var modtaget, hjalp til med at opdage fejl og mangler i dokumentationen, som efterfølgende blev rettet på et arbejdsmøde. 
De reviews, som skulle fremstilles, har givet mere inspiration til gruppens eget projekt, og givet kompetencer indenfor rettelse af tekniske dokumenter. At påpege fejl og mangler, i stedet for at komme med forslag til forbedring, gjorde reviewet mere objektivt til andres arbejde. Når der modtages review, holdes det på et neutralt plan, og der forsøges ikke at forsvare den skrevne dokumentation, men istedet skal den modtagende gruppe koncentrere sig om de modtagne kommentarer. Kommentarene kunne diskuteres på et efterfølgende møde, og de vigtige mangler og fejl kunne rettes i dokumentationen.

