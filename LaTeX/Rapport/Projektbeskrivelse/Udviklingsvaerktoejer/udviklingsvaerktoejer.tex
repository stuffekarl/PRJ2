\section{Udviklingsværktøjer}
Under denne sektion vil der blive gennemgået de forskellige udviklingsværktøjer som hovedsageligt er blevet anvendt under dette projekts design-, implementerings- og integrationsproces.

\subsection{Multisim}
Gruppen har valgt multisim til design og simulering af hardware. Styrkerne ved multisim er at skabe et overblik og muligheden for at simulere forskellige hardware moduler, med de ønskede komponenter, fra et rigt bibliotek. Svaghederne ved multisim er, at det nogle gange kan være svært at simulere et kredsløb korrekt samt i værste tilfælde, at multisim ikke har mulighed for at simulere en hardware komponent.\\
I dette projekt er multisim anvendt under design og implementeringsprocessen for hardware.

\subsection{Visual Studio 2013}
Dette program er et redskab til kodning og debugging af kildekode under implementering af software.\\
I dette projekt er Visual Studio anvendt under implementerings- og integrationsprocessen for kodning af PC softwaremoduler, i C++.
Visual Studio er valgt pga. det er brugt i undervisningen.

\subsection{Atmel Studio}
Atmel studio er en forgrening af Visual Studio 2010. Programmet er anvendt til kodning og debugging af microcontrollere under implementering af software. Det har senere under forløbet vist sig at det har vakt mindre problemer at kode microcontrollerne med dette værktøj, da det under software design blev bestemt at kode i C++. Atmel Studio har en stærk compiler til C programmering, men den er ikke lige så stærk til C++, og endte med at give nogle mindre problemer under software implementeringen.

I dette projekt er Atmel Studio anvendt under implementerings- og integrationsprocessen for kodning af microcontoller software moduler, i C++. Under arbejdet med microcontrollere blev der gjort erfaringer (se s. \pageref{P-sec:micro} i projektdokumentationen). %TODO Indsæt reference til dokumentation afsnit impl.