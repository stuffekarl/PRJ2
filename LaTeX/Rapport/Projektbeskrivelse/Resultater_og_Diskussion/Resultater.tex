\section{Resultater og diskussion}
Projektet er endt ud med en prototype af et produkt, der med få undtagelser, bestod accepttesten. Det blev valgt ikke at implementere TV- og radiodelen af produktet, da det blev bestemt at fokus skulle ligge på det resterende system. De problemer og udfordringer der opstod under gennemførelsen af projektet, er blevet løst via et samarbejde i gruppen, hvor alle bidrog med viden og faglige kundskaber. De punkter som der i kravsspecifikationen blev sat fokus på, er blevet implementeret og systemet er stabilt, fejlfrit og fungerer selvstændigt.

Igennem projektperioden er der for hardware blevet lagt vægt på løsninger, der uden at være alt for komplicerede, kan klare deres opgaver som beskrevet. For at sikre stabile logiske niveauer i systemet, og dermed undgå prel, blev det valgt at implementere operationsforstærkere som komparatore med både positiv og negativ spændingsforsyning. Under implementeringen af hardware stødte gruppen på en del problemer, med de designs der var lavet på forhånd. Men med den rette vejledning, forblev projektet på sporet og endte ud med at have en tilfredsstillende prototype. I forhold til hvad der blev forventet af designet, har implementeringsprocessen været meget lærerig, da de problemer der opstod var med til at vise nye perspektiver på kredsløbene. Alle komponenterne blev realiseret på veroboard og testet hver for sig i modultests.

For softwaredelen var det vigtigt at få designet og implementeret noget enkelt men funktionelt kode. Processen har ledt til forståelse af, hvor vigtigt det er at have en test klar før selve kodningen af softwaren foretages. Softwaregruppen fandt desuden ud af, at det skal gøres helt klart hvilken klasse der skal gøre hvad og hvornår, for at undgå misforståelser mellem klasserne. Ved softwareintegration blev de dele af koden der var mulige at teste, som fx transtmitter UART'en, grundigt testet. Dog var det i andre tilfælde nødvendigt at vente til den samlede integrationstest, for at opdage de fejl der var på hhv. transmitteren og receiveren. 
Der blev konstrueret en såkaldt ”sladrehanks klasse”, der kunne fortælle hvad der sker under kommunikation med UART og over elnettet.

Integrationstesten var et afgørende punkt for projektet, da det er første gang det hele skulle sættes sammen. Der blev hurtigt fundet nogle fejl, som blev løst efterhånden. Stille og roligt begyndte systemet at fungere, som det blev designet til. Samarbejdet mellem software og hardware fungerede som det skulle, takket være en god planlægning af grænsefladerne mellem elementerne i systemet. For at finde de fejl der opstod, blev der sat en langsommere frekvens ind, i stedet for $50Hz$, og relevante outputs blev vist på LED’er, således det var muligt at aflæse hvad der blev sendt og modtaget. 

Det hele endte ud med en accepttest, som systemet bestod næsten uden undtagelser. Da det relativt tidligt i forløbet blev valgt ikke at implementere radio- og TV-delen, var det ikke muligt at bestå de tests der var opstillet for disse punkter. Systemet endte ud med at blive fejlfrit og meget sikkert, da det også blev testet med en væsentligt højere frekvens end $50Hz$ og stadig fungerede. Helt fra starten blev det vedtaget, at formålet med projektet var et simpelt system, og fokus skulle lægges på indlæring. Denne beslutning har gruppen værdsat højt, og har gjort at hvert medlem i gruppen har fået mest muligt ud af projektet.
