\section{Fildeling}

\subsection{Dropbox}

Gennem projektet har Dropbox været den primære fildelings service til at holde styr på alt undtaget kildekode. Beslutningen skyldes at alle medlemmer i gruppen har erfaring med Dropbox, og det var nemt at sætte op. Dropbox indeholdte mødereferater, tidsplan, coaching, diagrammer samt andre bilag. 
Mappe-strukturen var en stor hjælp angående indeksering af tex-filer, og generelt har brugen af Dropbox gennem forløbet været uproblematisk. Desuden har Dropbox's restorefunktion som har reddet rapporten, da den blev fejlagtigt slettet.

\subsection{SVN}

Til deling af kildekode var overvejelserne mellem Git og SVN. Git var interessant, da flere virksomheder bruger dette, og det vil give noget relevant erfaring. SVN blev dog valgt til deling af kildekode, siden et gruppemedlem havde et SVN repository oppe på \url{riouxsvn.com}, samt ISE faget gav undervising i brugen af SVN. Sidst skal det fremhæves, at den klient vi valgte at bruge, TortoiseSVN\cite{lib:Tortoise}, har meget simpel tilgang til de mest væsentlige kommandoer, \texttt{update} og \texttt{commit}, mens git umidelbart havde en mere kompliceret tilgang. Så simplicitet var valgt over funktionalitet.

SVN blev kraftigt brugt gennem software implementeringsfasen, og endte med 88 revisioner. I starten var der lidt problemer med committing, dog blev dette løst hurtigt, efter folk blev bekendte med tjenesten.