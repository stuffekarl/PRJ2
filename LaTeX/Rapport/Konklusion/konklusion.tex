\chapter{Konklusion}
Overordnet set opfylder projektarbejdet fuldstændigt opgavens mål og formål inden for de rammer, der er givet i projektoplægget. 
Der er blevet designet og realiseret et system, som kan forebygge indbrud i en bolig. 
De dele af systemet, der er blevet realiseret, fungerer stort set fejlfrit og meget stabilt. 

Det besluttedes fra starten at systemet skulle være så simpelt som muligt, hvilket med sikkerhed har medvirket til det gode resultat. 
Heri ligger en af de vigtigste erfaringer i projektarbejdet. 
Denne beslutning gennemstyrer hele projekarbejdet og det gælder således, at såvel HW som SW er forsøgt designet og implementeret så simpelt som muligt.
Det er dog klart at man er nødt til at gå på kompromis med ekstra funktionalitet, fejlsikring mm., hvilket har stillet projektgruppen overfor nogle vigtige beslutninger undervejs i forløbet. 
Det er undervejs i forløbet fx ofte blevet nævnt, at ''...det ville have været rart med tovejskommunikation'' på elnettet. 
Havde vi implementeret dette, kunne vi have sikret os mod fejl ved simpelthen at sende en kommando igen, hvis den ikke blev modtaget. 
Vi har dog ikke oplevet decideret behov herfor; X.10 kommunikationen på elnettet fungerer fejlfrit.

En mindst ligeså vigtigt læring ligger inden for styring af processen. 
Vi har efter vores egen mening formået at styre projektforløbet særdeles godt. 
Der har undervejs kun været behov for at rette i tidsplanen enkelte gange, og dokumentation er blevet skrevet og opdateret løbende. 
Dette har betydet at der har været god tid i de sidste faser og god tid til at skrive rapport. 

Projektgruppen har udviklet sig meget under projektforløbet. 
Gruppen havde ikke arbejdet sammen som helhed før, og der var som forventeligt lidt udfordringer i starten.
Det var særdeles fornuftigt at starte med at lave en samarbejdsaftale, og derigennem få afstemt forventninger til arbejdet og systemet.  
Alle gruppemedlemmer var fra starten meget engagerede, hvilket betød at gruppen ofte måtte bruge en del ressourcer på at blive enige om forskellige beslutninger. 
Efterhånden som projektarbejdet er skredet frem, og gruppemedlemmerne har fundet deres roller i gruppen, er samarbejdet blevet mere og mere strømlinet. 
Der er stadig plads til yderligere udvikling, såfremt samarbejdet fortsættes.

\clearpage